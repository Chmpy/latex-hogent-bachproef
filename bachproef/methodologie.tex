%%=============================================================================
%% Methodologie
%%=============================================================================

\chapter{\IfLanguageName{dutch}{Methodologie}{Methodology}}%
\label{ch:methodologie}

In dit hoofdstuk wordt een overzicht gegeven van de methodologie die is toegepast om het privacybewuste spelling- en grammaticacontrole-toetsenbord voor Android te ontwikkelen. Het onderzoek is opgedeeld in verschillende fasen, waarbij elke fase specifieke doelstellingen, deliverables en onderzoeksmethoden omvat.

\subsection{Literatuurstudie}

Deze fase bestond uit een uitgebreide studie van bestaande literatuur op het gebied van Natural Language Processing (NLP), privacybewuste technologieën en taalmodellen zoals BERT en RobBERT. Deze studie legde de theoretische basis voor de verdere ontwikkeling van het toetsenbord.

\subsection{Modelselectie}

Op basis van de literatuurstudie werd RobBERT geselecteerd vanwege de effectiviteit in Nederlandse taaltaken en de mogelijkheid tot optimalisatie voor mobiele apparaten \autocite{Delobelle2020}. Alternatieve lichte modellen zoals DistilBERT en TinyBERT werden ook geëvalueerd \autocite{Sanh2019DistilBERT}.

\subsection{Model Conversie en Output Verwerking}

Het RobBERT-model werd geconverteerd van PyTorch naar ONNX en vervolgens naar TensorFlow Lite, gebruikmakend van PyTorch, ONNX en TensorFlow libraries. De methode om de output te verwerken werd besproken en de uitdagingen die bij het geheel kwamen kijken.

\subsection{Gebruikersinterface Ontwikkeling}

De ontwikkeling van een gebruikersinterface begon met een eigen ontwerp, maar uiteindelijk werd een fork van een bestaand open-source toetsenbord gebruikt. Dit zorgde voor een snellere implementatie en gebruik van beproefde functionaliteit.

\subsection{Integratie van Taalmodellen}

Het geoptimaliseerde RobBERT-model werd geïntegreerd in de Android-applicatie met behulp van TensorFlow Lite voor Android. Dit framework leek de nodige API's en tools voor modelimplementatie te beheren\autocite{ulukaya2023}.

\subsection{Deliverables}

Elke fase van het onderzoek leverde specifieke deliverables op:
\begin{itemize}
    \item Literatuurstudie: Grondige theoretische basis en literatuuroverzicht.
    \item Modelselectie: Evaluatie en keuze van het RobBERT-model.
    \item Modelconversie en optimalisatie: Geoptimaliseerd model in TensorFlow Lite/ONNX.
    \item Gebruikersinterface ontwikkeling: Gebruikersvriendelijke en privacybewuste interface, gebaseerd op een fork van een open-source toetsenbord.
    \item Integratie van taalmodellen: Werkende integratie van RobBERT in de Android-applicatie.
\end{itemize}

Deze methodologie biedt een systematisch plan van aanpak voor de ontwikkeling, implementatie en evaluatie van het privacybewuste toetsenbord, met een nadruk op nauwkeurigheid, gebruikersprivacy en efficiënte lokale NLP-verwerking.