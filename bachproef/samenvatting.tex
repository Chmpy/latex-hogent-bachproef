%%=============================================================================
%% Samenvatting
%%=============================================================================

% Een goede abstract biedt een kernachtig antwoord op volgende vragen:
%
% 1. Waarover gaat de bachelorproef?
% 2. Waarom heb je er over geschreven?
% 3. Hoe heb je het onderzoek uitgevoerd?
% 4. Wat waren de resultaten? Wat blijkt uit je onderzoek?
% 5. Wat betekenen je resultaten? Wat is de relevantie voor het werkveld?
%
% Daarom bestaat een abstract uit volgende componenten:
%
% - inleiding + kaderen thema
% - probleemstelling
% - (centrale) onderzoeksvraag
% - onderzoeksdoelstelling
% - methodologie
% - resultaten (beperk tot de belangrijkste, relevant voor de onderzoeksvraag)
% - conclusies, aanbevelingen, beperkingen
%
% LET OP! Een samenvatting is GEEN voorwoord!

%%---------- Nederlandse samenvatting -----------------------------------------
%
% TODO: Als je je bachelorproef in het Engels schrijft, moet je eerst een
% Nederlandse samenvatting invoegen. Haal daarvoor onderstaande code uit
% commentaar.
% Wie zijn bachelorproef in het Nederlands schrijft, kan dit negeren, de inhoud
% wordt niet in het document ingevoegd.

\IfLanguageName{english}{%
\selectlanguage{dutch}
\chapter*{Samenvatting}
\lipsum[1-4]
\selectlanguage{english}
}{}

%%---------- Samenvatting -----------------------------------------------------
% De samenvatting in de hoofdtaal van het document

\chapter*{\IfLanguageName{dutch}{Samenvatting}{Abstract}}

In deze scriptie wordt een privacybewust spelling- en grammaticacontrole-toet\-sen\-bord voor Android ontwikkeld, met gebruik van RobBERT en TensorFlow. Het centrale thema van dit onderzoek is de ontwikkeling van een toetsenbord dat nauwkeurige taalverwerking biedt en offline functioneert, zodat gebruikers volledige controle over hun persoonlijke gegevens behouden. De onderzoeksvraag richt zich op hoe een dergelijk toetsenbord kan worden ontwikkeld, met een focus op de ondersteuning van Nederlandse taal, de balans tussen nauwkeurigheid, privacy, en de schaalbaarheid naar andere talen en gebruikssituaties.
Het doel van dit onderzoek was een proof-of-concept te ontwikkelen voor een privacybewust toetsenbord, waarbij een geoptimaliseerde versie van het RobBERT-model wordt gebruikt voor offline taalverwerking. De verwachte resultaten omvatten een werkend prototype dat nauwkeurige spelling- en grammaticacontrole biedt zonder externe gegevensverwerking, volledige controle van de gebruiker over persoonlijke gegevens, flexibiliteit voor uitbreiding naar meerdere talen, en inzichten in de balans tussen nauwkeurigheid en privacy in taalverwerkingsmodellen.
De methodologie van dit onderzoek omvat verschillende fasen: een uitgebreide literatuurstudie naar bestaande NLP-technologieën en privacykwesties, de selectie van geschikte taalmodellen waarbij uiteindelijk gekozen is voor RobBERT 2023-dutch-large vanwege zijn prestaties in Nederlandse taaltaken, de conversie van dit model vanuit PyTorch naar TensorFlow Lite, geschikt voor mobiele apparaten, en de integratie van dit model in een bestaande open-source toetsenbord applicatie.
De resultaten van het onderzoek tonen aan dat het mogelijk is een werkend prototype te ontwikkelen dat nauwkeurige spelling- en grammaticacontrole biedt zonder externe gegevensverwerking, wat bijdraagt aan de privacy van gebruikers. Echter, het prototype bleek uiteindelijk geen praktisch succes vanwege de beperkingen van huidige libraries en conversie van het model. Hoewel het theoretisch mogelijk is gebleken, bestaan er in de praktijk nog aanzienlijke uitdagingen.
Concluderend draagt dit onderzoek bij aan de ontwikkeling van privacybewuste technologieën door gebruikers meer controle te geven over hun gegevens en tegelijkertijd geavanceerde taalverwerkingsfunctionaliteiten te bieden. Het succes in theorie kan de basis vormen voor verdere ontwikkeling en verbetering, wat de autonomie van gebruikers kan versterken in een steeds meer data-gedreven wereld.
